\documentclass[14pt,letterpaper]{extarticle}
\usepackage[utf8]{inputenc}
\usepackage{amsmath}
\usepackage{amsfonts}
\usepackage{amssymb}
\usepackage{fullpage}
\usepackage{marginnote}
\usepackage{enumitem}
\usepackage{graphicx}
\usepackage{wrapfig}
\usepackage{multirow}
\usepackage{tabularx}
\usepackage{array}
\usepackage{setspace}
\usepackage[top=0.25in,bottom=0.25in,left=0.4in,right=1.1in]{geometry}
\marginparwidth = 1in

\begin{document}

\pagestyle{empty}

\newcommand{\newsection}[1]{
\bigskip\medskip
\noindent\textbf{\large{#1}}
}

\newcolumntype{L}{>{\raggedright\arraybackslash}X}
\newcolumntype{R}{>{\raggedleft\arraybackslash}X}

\noindent
\begin{tabularx}{\linewidth}{@{}lR@{}}
\multirow{2}{*}{\noindent\textbf{\Large{Shawn Rigdon}}} & \footnotesize{828-550-3668} \\[-4pt]
& \footnotesize{shawn.patrick.rigdon@gmail.com} \\[-4pt]
\end{tabularx}

\smallskip
\noindent\rule[\baselineskip]{\linewidth}{1.5pt}


\noindent\textbf{\large{Overview}}

\medskip
\noindent\footnotesize{Engineer and startup founder with a track record of delivering real-time,
ML-driven systems across academia, industry, and entrepreneurial ventures. As CTO of Digi-Clone,
I lead the development of a virtual try-on application, bridging computer vision, 3D modeling,
and mobile deployment. Notable projects from my career include a mm-wave scanner for concealed
weapon detection and an embedded stereo camera for people tracking in retail applications. As a
research engineer at USC, I prototyped full sensor and software solutions for biometric fraud
detection. I specialize in imaging pipelines for real-time computer vision applications, and I am
experienced in embedded systems (hardware and software design and manufacturing), GPU acceleration,
and algorithm development.}

\newsection{Education}

\medskip\noindent
\begin{tabular}{@{}ll@{}}
\footnotesize{The Georgia Institute of Technology:} & \footnotesize{MS in Electrical and Computer Engineering}\\[5pt]
\footnotesize{Western Carolina University:} & \footnotesize{BS in Electrical Engineering}\\
\footnotesize{(\textit{magna cum laude})} & \footnotesize{BS in Applied Mathematics}\\
\end{tabular}
\marginnote{\footnotesize{\textit{2014}}}[-12pt]
\marginnote{\footnotesize{\textit{2009}}}[8pt]

\newsection{Professional Experience}

\newcommand{\experience}[4]{
\medskip
\marginnote{\footnotesize{\textit{#1}}}
\noindent\footnotesize{\textbf{#2} \textit{(#3)}}\\
#4
}

\experience{2025-Present}{Digi-Clone Ltd Co}{Founder and CTO}{
The mission of Digi-Clone is to automate the creation of realistic digital-twin human avatars, enabling
broad adoption of their applications. The focus application for the company is an interactive, 3D virtual
fitting room, where users can dress a custom avatar in a simulated environment. The app allows users to
visualize the way garments fit and move on their unique, virtual body prior to purchase. As founder and CTO,
I have responsibilities in all areas of the business, including leading the product development. The development
involves three main challenges: 1) reconstructing a rigged body model from a minimal set of pose-constrained
photos captured by a consumer smartphone, 2) building a large database of 3D garment models representative of
retail inventory, and 3) registration of the 3D garments with arbitrary body meshes. My cofounder and I have
received the program director’s recommendation for the National Science Foundation’s SBIR Phase I award in
Augmented, Virtual, and Mixed Reality. We are currently awaiting the final award decision, which will fund our
development activities for the next year.
}

\experience{2020-Present}{Computational Modeling and Software Services, LLC}{Owner}{
Helping startup tech businesses implement solutions to complex computational problems requiring
automated decision making from real-time data streams. As a consultant, I offer assistance on the
full pipeline of real-time system software design including hardware selection and interfacing,
data acquisition and processing, machine learning techniques, and system deployment at scale.
}

\experience{2020-2023}{USC Information Sciences Institute}{Lead Research Engineer}{
Lead engineer of several projects in the Visual Intelligence and Multimedia Analytics Laboratory (VIMAL).
This is a small research team working on machine learning techniques for biometric security and fraud detection
in images and video. The role involves a myriad of engineering tasks focused on developing sensor platforms for
deploying the machine learning algorithms in a real-time environment. Responsibilities include software
architecture, computational optimizations, data acquisition from multiple sensing modalities, sensor hardware
prototyping (some circuit design and CAD of a 3D printed housing), and data collection design and operation.
The position also requires writing proposals for large research grants and designing IRB approved human subject
research trials.
}

\experience{2019-2020}{Liberty Defense Technologies, Inc.}{Sr. Software Engineer}{
Software architect of real-time microwave imaging system for concealed weapon detection. The position was mainly
responsible for reconstructing and processing images from data given by a short range mm wave radar transceiver.
This required integration of several hardware components including a stereo camera used to locate subjects in the
scene.  To achieve a real-time frame rate, many of the algorithms were written using CUDA to parallelize processing
on an onboard Nvidia GPU.  Other responsibilities included implementing fast algorithms for person/object
segmentation, triggering captures and other timing between camera and microwave images, and generating training
data for object detection models.
}

\experience{2015-2019}{FLIR Systems, Inc.}{Software Engineer}{
Firmware/software development of embedded stereo cameras for people tracking in a retail environment.
Application areas include: image processing, traffic counting, employee/customer classification using
vision and inertial sensor data fusion, BLE beacon advertising. Designed automated mfg. processes
(e.g. HW functionality testing, focusing and calibration of stereo lenses) to optimize product scalability.
}

\pagebreak
\experience{2010-2013}{TekTone Sound and Signal Mfg., Inc.}{Hardware Engineer}{
Designed embedded hardware for nurse call products deployed in hospitals and assisted living facilities.
Designed the first UL listed wireless nurse call system.
}

%\experience{2010}{Space and Naval Warfare Systems Command}{Project Engineer}{
%Project manager overseeing contracted air traffic control systems installations.
%}
%
%\experience{2008-2010}{Haywood Electric Membership Corporation}{Systems Engineer Intern}{
%Substation installation and debugging of TWACS, an automated meter reading system.
%Led company wide adoption of North Carolina One Call System (an online database application for
%locating underground wire).
%}

\newsection{Teaching Experience}

\medskip\noindent
\begin{tabularx}{\linewidth}{@{}lL@{}}
\footnotesize{The Georgia Institute of Technology} & \footnotesize{Circuit Analysis (ECE2040)}\\
\marginnote{\footnotesize{\textit{2013-2014}}}[-11pt]
\footnotesize{\textit{Teaching Assistant}} & \footnotesize{TESSAL (Teaching Enhancement via Small Scale Affordable Labs) Center}\\
& \footnotesize{Fundamentals of Digital System Design
(ECE2020)}\\
\end{tabularx}

%\pagebreak
\newsection{Software Tools}

\medskip\noindent
\begin{tabularx}{\linewidth}{@{}lL@{}}
\footnotesize{Languages:} & \footnotesize{C, C\texttt{++}, Python}\vspace{0.5mm}\\
\footnotesize{Tools and Utilities:} & \footnotesize{MATLAB, CUDA, TensorFlow, PyTorch, OpenCV, Docker, Git, Subversion,}\\
                                    & \footnotesize{Package Managers (opkg, dpkg, pacman), Neovim, Visual Studio, Qt,}\\
                                    & \footnotesize{\LaTeX, PADS PCB Layout, OrCAD Schematic Capture, Fusion 360}\vspace{0.5mm}\\
\footnotesize{Operating Systems:}   & \footnotesize{Microsoft Windows, Linux}\\
\end{tabularx}

\newsection{Publications}\vspace{5.3pt}\\
\noindent
\footnotesize{M. Woldeyohannes, J. Schenk, R. Ingel, S. Rigdon, M. Pate, J. Graham, M. Clare, W. Yang, M. Fiddy, ``Internal Field Distribution Measurement in 1-D Strongly Anisotropic Sub-Wavelength Periodic Structures of Finite Length,'' Optics Express Vol. 19, Issue 1 (Jan. 3, 2011) pp: 81-92.}

\newsection{Presentations}\vspace{5.3pt}\\
\footnotesize{``Integration of Vision and BLE Transmissions for Employee Recognition,'' FLIR Research Symposium, FLIR Systems, Inc.}
\marginnote{\footnotesize{\textit{2018}}}[-12pt]

\medskip\noindent
\footnotesize{``Enormous  Field  Enhancement  in  Periodic  Birefringent Media,''
National Conference of Undergraduate Research, University of Wisconsin, LaCrosse}\marginnote{\footnotesize{\textit{2009}}}[-12pt]

\medskip\noindent
\footnotesize{``Analysis of Force Controlled Arm Movement,'' Undergraduate Expo, Western Carolina University}
\marginnote{\footnotesize{\textit{2009}}}

\newsection{Honors and Awards}\vspace{5.3pt}\\
\noindent\footnotesize{President's Fellowship Recipient, The Georgia Institute of Technology}
\marginnote{\footnotesize{\textit{2013}}}

\medskip
\noindent\footnotesize{Most Outstanding Senior in Electrical Engineering, Western Carolina University}
\marginnote{\footnotesize{\textit{2009}}}

\medskip
\noindent\footnotesize{Freshman Math Award, Western Carolina University}
\marginnote{\footnotesize{\textit{2005}}}


\end{document}
